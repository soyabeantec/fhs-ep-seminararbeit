\documentclass[12pt]{article}
\setlength{\headheight}{14.49998pt}
\addtolength{\topmargin}{-2.49998pt}

% Packages
\usepackage[ngerman]{babel} % Setting the document language to German
\usepackage[autostyle=true, german=quotes]{csquotes} % Setting the quotes style to German

\usepackage[authordate,backend=biber]{biblatex-chicago} % Chicago style citations
\usepackage{fancyhdr}
\usepackage{geometry} % For setting page dimensions
\geometry{a4paper} % Sets the paper size to A4

% Bibliography
\addbibresource{FHS_EP_Sokrates-Nichtwissen_u_kuenstliche_Intelligenz.bib}

% Header and footer style
\pagestyle{fancy}
\fancyhf{} % clear all header and footer fields
\fancyhead[L]{II$<$II} % R for right side, L for left, C for centre
\fancyfoot[C]{\thepage}

\begin{document}
\pagestyle{fancy}

% Title page
\title{Sokrates' Nichtwissen und künstliche Intelligenz - Ein Versuch des Dialogs}
\author{Emanuel Klatzer}
\date{Wintersemester 2023/2024}
\maketitle % Generates title page
\thispagestyle{fancy} % Sets the custom header/footer for the first page


\section{Einleitung}
In der folgenden Arbeit soll ein Versuch unternommen werden, zwei scheinbar unvereinbare Themen
miteinander in Dialog treten zu lassen. Einerseits soll das Nichtwissen des Sokrates, wie es in
Platons Dialogen dargestellt wird, und andererseits die künstliche Intelligenz (KI), wie sie in
der Informatik und der Philosophie diskutiert wird, miteinander in Dialog treten. Dabei soll die 
Frage im Zentrum stehen, ob und inwiefern die KI von Sokrates' Nichtwissen lernen kann und ob das
Nichtwissen des Sokrates als Vorbild für die KI und ihre Entwicklung dienen kann.

Als Fundament und Ausgangspunkt wollen wir uns dabei auf Platons Dialoge \textit{Apologie},
\textit{Meno} und \textit{Theaitetos} stützen, um Sokrates' Nichtwissen zu verstehen. Sokrates'
Denken - welches uns hauptsächlich durch Platons Dialoge überliefert ist - schreibt genau
diesem Nichtwissen eine zentrale Rolle zu, wenn es darum geht Schritt für Schritt in Richtung
wahrer Erkenntnis zu wandern. Dieses Nichtwissen des Sokrates soll uns im Verlauf dieser Arbeit als
Grundlage für eine tiefgründigere Untersuchung und einen Dialog zwischen antiker Philosophie und
moderner KI-Überlegungen dienen.

Parallel dazu wollen wir uns dem Thema der künstlichen Intelligenz aus der Perspektive des Philosophen
Gotthard Günther nähern. Günther, der sich in den 1950er und 1960er Jahren intensiv mit der Kybernetik
und der KI beschäftigte, hat in seinen Schriften eine eigene Sichtweise auf die KI entwickelt, die
sich von anderen zeitgenössischen Ansätzen unterscheidet. Günther vertritt die These, dass die KI
nicht als bloße Nachahmung des menschlichen Denkens verstanden werden kann, sondern dass sie eine
eigene Form des Denkens darstellt, die sich vom menschlichen Denken unterscheidet. Diese These
wollen wir im Verlauf dieser Arbeit genauer untersuchen und in Dialog mit Sokrates' Nichtwissen
treten lassen. Im speziellen wollen wir uns dabei auf Günthers Schrift \textit{Das Bewusstsein der 
Maschinen - Eine Metaphysik der Kybernetik} stützen.

\section{Sokrates und das Konzept des Nichtwissens}

Das Konzept des Nichtwissens, oder auch anders ausgedrückt das Bewusstsein über die Grenzen des 
eigenen Wissens steht im Zentrum von Sokrates' Philosophie. Und dieses können wir in zahlreichen Dialogen
Platons nachvollziehen. In Platons \textit{Apologie} beispielsweise, in der Sokrates sich vor dem 
Gericht verantworten muss, sagt er: Ich weiß, dass ich nichts weiß. (Platon, Apologie 22d) \parencite[Apologie 22d]{platon-apologie}


\nocite{*}
\printbibliography % Generates the bibliography

\end{document}